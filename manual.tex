% !TEX TS-program = pdflatex
% !TEX encoding = UTF-8 Unicode

% This is a simple template for a LaTeX document using the "article" class.
% See "book", "report", "letter" for other types of document.

\documentclass[11pt]{article} % use larger type; default would be 10pt

\usepackage[utf8]{inputenc} % set input encoding (not needed with XeLaTeX)

%%% Examples of Article customizations
% These packages are optional, depending whether you want the features they provide.
% See the LaTeX Companion or other references for full information.

%%% PAGE DIMENSIONS
\usepackage{geometry} % to change the page dimensions
\geometry{a4paper} % or letterpaper (US) or a5paper or....
% \geometry{margin=2in} % for example, change the margins to 2 inches all round
% \geometry{landscape} % set up the page for landscape
%   read geometry.pdf for detailed page layout information

\usepackage{graphicx} % support the \includegraphics command and options

% \usepackage[parfill]{parskip} % Activate to begin paragraphs with an empty line rather than an indent

%%% PACKAGES
\usepackage{booktabs} % for much better looking tables
\usepackage{array} % for better arrays (eg matrices) in maths
\usepackage{paralist} % very flexible & customisable lists (eg. enumerate/itemize, etc.)
\usepackage{verbatim} % adds environment for commenting out blocks of text & for better verbatim
\usepackage{subfig} % make it possible to include more than one captioned figure/table in a single float
% These packages are all incorporated in the memoir class to one degree or another...

%%% HEADERS & FOOTERS
\usepackage{fancyhdr} % This should be set AFTER setting up the page geometry
\pagestyle{fancy} % options: empty , plain , fancy
\renewcommand{\headrulewidth}{0pt} % customise the layout...
\lhead{}\chead{}\rhead{}
\lfoot{}\cfoot{\thepage}\rfoot{}

%%% SECTION TITLE APPEARANCE
\usepackage{sectsty}
\allsectionsfont{\sffamily\mdseries\upshape} % (See the fntguide.pdf for font help)
% (This matches ConTeXt defaults)

%%% ToC (table of contents) APPEARANCE
\usepackage[nottoc,notlof,notlot]{tocbibind} % Put the bibliography in the ToC
\usepackage[titles,subfigure]{tocloft} % Alter the style of the Table of Contents
\renewcommand{\cftsecfont}{\rmfamily\mdseries\upshape}
\renewcommand{\cftsecpagefont}{\rmfamily\mdseries\upshape} % No bold!


\newcommand{\code}{}

%%% END Article customizations

%%% The "real" document content comes below...

\title{I-EPOS Manual}
\author{Peter Pilgerstorfer}
%\date{} % Activate to display a given date or no date (if empty),
         % otherwise the current date is printed 

\begin{document}
\maketitle

\section{Introduction}

\subsection{Install and setup}
TODO: netbeans/eclipse instructions
TODO: add libraries

\subsection{Execute the sample simulation}
TODO: run GUI
TODO: run from code: SimpleExperiment.main

\section{Architecture}
TODO: software architecture

\section{Use cases}
\subsection{Configure the simulation}
Let's take a look at the provided sample experiment \code{SimpleExperiment}. It contains a main function that specifies all parameters, starts the simulation and presents the results.

\subsubsection*{Dataset}
Initially the dataset is specified. Note that the \code{Dataset} object is not required for the algorithm, but it simplifies the configuration. Technically all we need is a way to get a list of possible plans for each agent. The \code{Dataset} interface provides this functionality via \code{Dataset.getPlans(int agentIdx)}.
There are two classes of datasets implemented:
\begin{itemize}
	\item \code{FileVectorDataset} is a dataset that is read from disk. Only the dataset folder has to be specified. The example datasets are located in the directory '<project dir>/input-data'. Section \ref{sec:new_dataset} describes the input format for this kind of dataset.
	\item \code{GaussianDataset} is a generated dataset where every plan is a vector drawn from a gaussian distribution. The parameters specify the number and dimensionality of the plans as well as mean and standard deviation of the distribution.
\end{itemize}
The number of agents can be set arbitrarily. However, when using a \code{FileVectorDataset}, the number of agents has to be below \code{FileVectorDataset.getNumAgents()}.

\subsubsection*{Cost functions}
The global cost function describes what we want to minimize globally. The local cost function describes what each agent wants to minimize locally. \code{lambda} is the tradeoff between global and local minimization. \code{lambda = 0} means only global cost is minimized, \code{lambda = 1} means only local cost is minimized.
For global cost functions we can choose any implementation of the interface \code{CostFunction} in general. However, for gradient descent based algorithms an instance of \code{DifferentiableCostFunction} is required. A list of possible cost functions is as follows:
\begin{itemize}
	\item \code{DotCostFunction} minimizes the dot product of a given vector with the global response. See Section \ref{sec:read_vec} how to read a vector from a file. Be aware that the dimensionality of the vector has to match the dimensionality of the dataset.
An example use case for this cost function is minimizing monetary cost. The agent plans contain the amount of resources they consume, and the vector passed to \code{DotCostFunction} describes the price for each resource.
As this is a linear cost function, I-EPOS always finds the optimal value in the first iteration.
	\item \code{SqrDistCostFunction} minimizes the (squared) distance of the global response to a given vector. See Section \ref{sec:read_vec} how to read a vector from a file. Be aware that the dimensionality of the vector has to match the dimensionality of the dataset.
This cost function tries to make the global response as similar to the provided target vector as possible.
	\item \code{VarCostFunction} minimizes the variance of the global response. Therefore it can be used to stabilize resource consumption over time, or for load balancing applications.
	\item \code{StdDevCostFunction} minimizes the standard deviation of the global response. For I-EPOS there is no difference between minimizing standard deviation and minimizing variance, as the functions share the same minima.
	\item \code{MaxCostFunction} (non-differentiable) minimizes the maximum value of the global response. This function is useful for peak reduction.
\end{itemize}
Local cost functions have to implement the \code{PlanCostFunction} interface. Two functions are implemented:
\begin{itemize}
	\item \code{IndexCostFunction} lets agents select plans with a small index. Therefore the plans in a dataset should be ordered in a way that lists preferable plans first.
	\item \code{PlanScoreCostFunction} lets agents select plans with a small score. The dataset specifies the score for each plan. See Section \ref{sec:new_dataset} for details how to specify this information in a dataset.
\end{itemize}

\subsubsection*{Network topology}


        // network
        int numChildren = 2; // +- of using this topology

        // algorithm
        int numIterations = 20;
        PlanSelector<IeposAgent<Vector>, Vector> planSelector = new IeposPlanSelector();

        // loggers
        LoggingProvider<IeposAgent<Vector>> loggingProvider = new LoggingProvider<>();
        loggingProvider.add(new GlobalCostLogger());
        loggingProvider.add(new LocalCostLogger());
        loggingProvider.add(new TerminationLogger());
        loggingProvider.add(new JFreeChartLogger());
        loggingProvider.add(new GraphLogger<>(GraphLogger.Type.Change, null));

        // start experiment
        new SimpleExperiment().run(
                numChildren,
                numIterations,
                numAgents,
                agentIdx -> {
            List<Plan<Vector>> possiblePlans = dataset.getPlans(agentIdx);
            AgentLoggingProvider agentLP = loggingProvider.getAgentLoggingProvider(agentIdx, 0);

            IeposAgent newAgent = new IeposAgent(numIterations, possiblePlans, globalCostFunc, localCostFunc, agentLP, random.nextLong());
            newAgent.setLambda(lambda);
            newAgent.setPlanSelector(planSelector);
            return newAgent;
        });

        loggingProvider.print();


FileVectorDataset -> getNumAgents()

\subsection{Write a new optimization function} \label{sec:new_func}
\subsection{Add a new dataset} \label{sec:new_dataset}

\appendix{}
\subsection{Glossary}
\subsection{Utility functions}
\subsubsection{Reading a vector} \label{sec:read_vec}
A vector can be read from a file via \code{VectorIO.readVector(File vectorFile)}. The file is assumed to be text file, containing a comma-separated list of double values that make up the vector.
\end{document}
