% !TEX TS-program = pdflatex
% !TEX encoding = UTF-8 Unicode

% This is a simple template for a LaTeX document using the "article" class.
% See "book", "report", "letter" for other types of document.

\documentclass[11pt]{article} % use larger type; default would be 10pt

\usepackage[utf8]{inputenc} % set input encoding (not needed with XeLaTeX)

%%% Examples of Article customizations
% These packages are optional, depending whether you want the features they provide.
% See the LaTeX Companion or other references for full information.

%%% PAGE DIMENSIONS
\usepackage{geometry} % to change the page dimensions
\geometry{a4paper} % or letterpaper (US) or a5paper or....
% \geometry{margin=2in} % for example, change the margins to 2 inches all round
% \geometry{landscape} % set up the page for landscape
%   read geometry.pdf for detailed page layout information

\usepackage{graphicx} % support the \includegraphics command and options

% \usepackage[parfill]{parskip} % Activate to begin paragraphs with an empty line rather than an indent

%%% PACKAGES
\usepackage{booktabs} % for much better looking tables
\usepackage{array} % for better arrays (eg matrices) in maths
\usepackage{paralist} % very flexible & customisable lists (eg. enumerate/itemize, etc.)
\usepackage{verbatim} % adds environment for commenting out blocks of text & for better verbatim
\usepackage{subfig} % make it possible to include more than one captioned figure/table in a single float
% These packages are all incorporated in the memoir class to one degree or another...

%%% HEADERS & FOOTERS
\usepackage{fancyhdr} % This should be set AFTER setting up the page geometry
\pagestyle{fancy} % options: empty , plain , fancy
\renewcommand{\headrulewidth}{0pt} % customise the layout...
\lhead{}\chead{}\rhead{}
\lfoot{}\cfoot{\thepage}\rfoot{}

%%% SECTION TITLE APPEARANCE
\usepackage{sectsty}
\allsectionsfont{\sffamily\mdseries\upshape} % (See the fntguide.pdf for font help)
% (This matches ConTeXt defaults)

%%% ToC (table of contents) APPEARANCE
\usepackage[nottoc,notlof,notlot]{tocbibind} % Put the bibliography in the ToC
\usepackage[titles,subfigure]{tocloft} % Alter the style of the Table of Contents
\renewcommand{\cftsecfont}{\rmfamily\mdseries\upshape}
\renewcommand{\cftsecpagefont}{\rmfamily\mdseries\upshape} % No bold!


\newcommand{\code}{}

%%% END Article customizations

%%% The "real" document content comes below...

\title{I-EPOS Manual}
\author{Peter Pilgerstorfer}
%\date{} % Activate to display a given date or no date (if empty),
         % otherwise the current date is printed 

\begin{document}
\maketitle

\section{Introduction}

\subsection{Install and setup}
TODO: netbeans/eclipse instructions
TODO: add libraries

\subsection{Execute the sample simulation}
TODO: run GUI
TODO: run from code: SimpleExperiment.main

\section{Architecture}
TODO: software architecture

\section{Use cases}
\subsection{Configure the simulation}
Let's take a look at the provided sample experiment \code{SimpleExperiment}. It contains a main function that specifies all parameters, starts the simulation and presents the results.

\subsubsection*{Dataset}
Initially the dataset is specified. Note that the \code{Dataset} object is not required for the algorithm, but it simplifies the configuration. Technically all we need is a way to get a list of possible plans for each agent. The \code{Dataset} interface provides this functionality via \code{Dataset.getPlans(int agentIdx)}.
There are two classes of datasets implemented:
\begin{itemize}
	\item \code{FileVectorDataset} is a dataset that is read from disk. Only the dataset folder has to be specified. The example datasets are located in the directory '<project dir>/input-data'. Section \ref{sec:new_dataset} describes the input format for this kind of dataset.
	\item \code{GaussianDataset} is a generated dataset where every plan is a vector drawn from a gaussian distribution. The parameters specify the number and dimensionality of the plans as well as mean and standard deviation of the distribution.
\end{itemize}
The number of agents can be set arbitrarily. However, when using a \code{FileVectorDataset}, the number of agents has to be below \code{FileVectorDataset.getNumAgents()}.

\subsubsection*{Cost functions}
The global cost function describes what we want to minimize globally. The local cost function describes what each agent wants to minimize locally. \code{lambda} is the tradeoff between global and local minimization. \code{lambda = 0} means only global cost is minimized, \code{lambda = 1} means only local cost is minimized.
For global cost functions we can choose any implementation of the interface \code{CostFunction} in general. However, for gradient descent based algorithms an instance of \code{DifferentiableCostFunction} is required. A list of possible cost functions is as follows:
\begin{itemize}
	\item \code{DotCostFunction} minimizes the dot product of a given vector with the global response. See Section \ref{sec:read_vec} how to read a vector from a file. Be aware that the dimensionality of the vector has to match the dimensionality of the dataset.
An example use case for this cost function is minimizing monetary cost. The agent plans contain the amount of resources they consume, and the vector passed to \code{DotCostFunction} describes the price for each resource.
As this is a linear cost function, I-EPOS always finds the optimal value in the first iteration.
	\item \code{SqrDistCostFunction} minimizes the (squared) distance of the global response to a given vector. See Section \ref{sec:read_vec} how to read a vector from a file. Be aware that the dimensionality of the vector has to match the dimensionality of the dataset.
This cost function tries to make the global response as similar to the provided target vector as possible.
	\item \code{VarCostFunction} minimizes the variance of the global response. Therefore it can be used to stabilize resource consumption over time, or for load balancing applications.
	\item \code{StdDevCostFunction} minimizes the standard deviation of the global response. For I-EPOS there is no difference between minimizing standard deviation and minimizing variance, as the functions share the same minima.
	\item \code{MaxCostFunction} (non-differentiable) minimizes the maximum value of the global response. This function is useful for peak reduction.
\end{itemize}
Local cost functions have to implement the \code{PlanCostFunction} interface. Two functions are implemented:
\begin{itemize}
	\item \code{IndexCostFunction} lets agents select plans with a small index. Therefore the plans in a dataset should be ordered in a way that lists preferable plans first.
	\item \code{PlanScoreCostFunction} lets agents select plans with a small score. The dataset specifies the score for each plan. See Section \ref{sec:new_dataset} for details how to specify this information in a dataset.
\end{itemize}

\subsubsection*{Network}
The network is considered to be a balanced tree with the same number of children for each inner node. The number of children for inner nodes can be specified. Be aware that the runtime of I-EPOS is exponential in the number of children. Luckily the optimization performance of a binary tree is already close to optimal in practice.

\subsubsection*{Logging}
Next we specify what information we want to gather from the simulation. For this task we specify a \code{LoggingProvider<A>}, where \code{A} is the class of the agent that is used in the simulation.
We then specify all information we want to log by adding \code{AgentLoggingProvider}s. Each \code{AgentLoggingProvider} is responsible for reading and presenting one specific type of data. The output is presented after the simulation by calling the method \code{LoggingProvider.print}. The following loggers are ready to use:
\begin{itemize}
	\item \code{GlobalCostLogger} logs the global cost in each iteration and prints the global cost for each individual iteration averaged over multiple simulations. Note that the sample shown in \code{SimpleExperiment} only performs one simulation. Multiple simulations can be performed with e.g. different seeds for the agents or different datasets. The only requirement is that the same \code{LoggingProvider} is used.
	\item \code{LocalCostLogger} logs the average local cost in each iteration and prints the average local cost for each individual iteration averaged over multiple simulations.
	\item \code{TerminationLogger} logs how many iterations it took for the algorithm to terminate. The algorithm is considered terminated if nothing changes between two consecutive iterations.
	\item \code{ProgressLogger} prints symbols every few iterations in order to show how far the algorithm has proceeded. It is only useful for large simulations.
	\item \code{JFreeChartLogger} shows a plot with global and (optionally) local cost values for each iteration in a new window. The logger requires \code{GlobalCostLogger} to be added to the \code{LoggingProvider} as well. If the \code{LocalCostLogger} is present, the local cost is also shown in the plot.
	\item \code{GraphLogger} shows a graph of the network at a given iteration in a new window. With the arrow keys you can switch between different iterations. Each agent is represented in a certain color. The color code depends on the specified type \code{GraphLogger.Type}. The following types are available:
	\begin{itemize}
		\item \code{Change} marks each agent that changed its selection in the previous iteration as black and all agents without change as white.
		\item \code{Index} colors each agent based on the index of the selected plan. Agents that selected the plan with minimal index are colored white and agents that selected the plan with maximal index are colored black.
	\end{itemize}
	\item \code{FileWriter} writes the log to the specified directory once \code{LoggingProvider.print()} is executed.
	\item \code{FileReader} reads the log from the specified directory once \code{LoggingProvider.print()} is executed. This logger can be used to show results from a previous simulation that were stored with \code{FileWriter}. A sample application of \code{FileReader} can be seen in \code{ReplayExperiment}.
\end{itemize}

\subsubsection*{Algorithm}
Finally, we have to specify the optimization algorithm. The algorithm is determined by the type of \code{Agent} that is used.
\begin{itemize}
	\item \code{IeposAgent} has quite a few options that were part of the research. We need to specify the number of iterations the algorithm should perform. For problems with less than 1000 agents the (local) optimum is usually found with less than 20 iterations. In addition we can specify a \code{PlanSelector}. The default is \code{IeposPlanSelector}. As part of the research that was done for I-EPOS, gradient descent motivated plan selectors were also developed. However, they are in inferior to the default in terms of optimization performance.
	\item \code{CohdaAgent} is an algorithm that was used as a baseline for I-EPOS. We only need to specify the number of iterations for this algorithm. Limitations: First, \code{CohdaAgent} does not support local cost. Therefore \code{LocalCostLogger} cannot be used. Second, the algorithm starts with an incomplete global response that is missing data from some agents. It takes \code{log(numAgents)/log(numChildren)} iterations for the global response to be complete. Third, even though COHDA does not require the network to be a tree, only tree networks can be simulated with this software.
\end{itemize}

\subsection{How to store evaluation results} \label{sec:store_results}
Evaluation results can be stored using the \code{LoggingProvider} \code{FileWriter}. Once the print command is executed on the \code{LoggingProvider} instance, the log is written to the specified file. With \code{FileReader}, the written file can be read again. See \code{ReplayExperiment} as an example how to read a log file.

\subsection{Write a new cost function} \label{sec:new_func}
Write a new class that extends the abstract class \code{CostFunction<DT>} or \code{DifferentiableCostFunction<DT>} where \code{DT} is the datatype that this cost function should operate on. The function \code{CostFunction.calcCost(DT value)} should compute the cost of the given value. For differentiable functions we also need to implement \code{DifferentiableCostFunction.calcGradient(DT value)} that should return the gradient of the function at point \code{value}.

\subsection{Add a new dataset} \label{sec:new_dataset}
One way of adding a new dataset is to use the existing class \code{FileVectorDataset} to read a custom dataset from the dataset directory. The dataset is a directory containing one file for each agent. The files should be named \code{agent\_<id>.plans}, where \code{<id>} is the id of the agent, starting from 0 upwards.
Each file should be a text file containing one row for each possible plan the agent can choose. A plan has the following layout: \code{<score>:<vector>}. The score is a double value that describes the cost this plan imposes for an agent. It can be used for local cost minimization\footnote{Set \code{lambda=0} if the score should be ignored.}. The vector is a comma separated list of double values.

It is also possible to code a new dataset. The only requirement for the dataset is to generate a list of plans given the index of an agent. \code{Dataset} is a handy interface that can be used to implement a new dataset. While the sample datasets all use vectors as datatype, the new dataset can use a custom datatype.

\subsection{Add a new datatype}
Write a new class that implements the interface \code{Datatype}. The functions of the interface \code{DataType} were designed work for vectors. So when implementing those functions, be aware that the semantic should be as it would be for vectors.

To use the new datatype, we also need to implement a new dataset and a new cost function that can handle the new datatype.

%\subsection{Add a new logger}
%Write a new class that extends the abstract class \code{AgentLogger}. Let us have a look what happens with the loggers that are added to \code{LoggingProvider} in order to understand what should be implemented:
%\begin{enumerate}
%	\item The object that is passed to the \code{LoggingProvider} in the experiment configuration is cloned once for each agent. During the simulation, an \code{AgentLogger} object is therefore only responsible for logging values of a single agent.
%	\item Before the algorithm starts, the method \code{init(agent)} is called. This is the place where the logger can be initialized with data dependent on the concrete agent that should be logged by this logger.
%	\item After each iteration of the simulation, the method \code{log(measurementLog, epoch, agent)} is called. We can write information into the \code{measurementLog}:
%	\code{measurementLog.log(epoch, agent.get, measuredValue)}
%\end{enumerate}

\appendix{}
\subsection{Glossary}
\subsection{Utility functions}
\subsubsection{Reading a vector} \label{sec:read_vec}
A vector can be read from a file via \code{VectorIO.readVector(File vectorFile)}. The file is assumed to be text file, containing a comma-separated list of double values that make up the vector.
\end{document}
